\documentclass[a4paper,12pt]{article}
\usepackage[utf8x]{inputenc}
\usepackage[english,russian]{babel}
\usepackage{amsmath}
\usepackage{amsbsy}
\usepackage{amsfonts}
\usepackage[	left=2cm,right=2cm, 
		top=2cm,bottom=2cm,
		bindingoffset=0cm]
			{geometry}
\begin{document}
\begin{flushleft}
\parskip=2ex
\section{Счётность множества рациональных чисел, несчётность множества действительных чисел.}

	\textbf{Утв.} Множество $\mathbb{Q}$ - счётно.
	
	\textbf{Док-во.} Все рациональные числа находятся в таблице, которую можно обойти "зигзагом".

	\textbf{Утв.} Множество $\mathbb{R}$ - несчётно.
	
	\textbf{Док-во.} Отрезок $[0; 1]$ - равномощный всему множеству $\mathbb{R}$. 
	Предположим, нам удалось сопоставить все числа из него натуральным. 
	Взяв новое число, отличающееся от $n$-ого в $n$-ом знаке, получим новое число, лежащее в этом отрезке, но не пронумерованное нами.
	Противоречие предположению. 

\section{Теорема о существовании (точной) верхней (нижней) грани множества.}

	\textbf{Утв.} Если $\mathbb{X}$ - ограниченное сверху (снизу) множество, то существует его точная врехняя (нижняя) грань.

	\textbf{Док-во.} Пусть $\mathbb{Y}$ - множество верхних граней $\mathbb{X}$. Так как $\mathbb{X}$ - ограничено, то $\mathbb{Y}$ - не пусто.
	Так как $\forall x \in \mathbb{X}, \forall y \in \mathbb{Y}: x \leq y$, то по свойству полноты действительных чисел $\exists c: \forall x \in \mathbb{X}, \forall y \in \mathbb{Y}: x \leq c \leq y$. Это и есть точная верхняя грань множества $\mathbb{X}$.
	
	\section{Бесконечно малые последовательности и их свойства.}

	\textbf{Опр.} Последовательность $\{x_n\}$ называется бесконечно малой (б.м) если 
	$$\lim_{x \to \infty} \{x_n\} = 0$$
	
	\textbf{Утв.} Алгебраическая сумма конечно числа б.м. последовательностей - б.м. последовательность.

	\textbf{Утв.} Произведение б.м. последовательности на ограниченную - б.м. последовательность.
	
	\section{Свойства пределов, связанные с неравенствами.}

	\textbf{Утв.} Сходящаяся последовательность ограничена.

	\textbf{Утв.} Справедливы следующие утверждения:
	\begin{enumerate}
	\item Если $$\exists N \in \mathbb{N}: \forall n > N: a_n \leq b_n \leq c_n$$ и $$\exists \lim_{n \to \infty} a_n = A, \exists \lim_{n \to \infty} c_n = A$$ то $$\exists \lim_{x \to \infty} b_n = A$$.
	\item Если $$\lim_{n \to \infty} a_n = a$$ и $$b > a$$ то $$\exists N \in \mathbb{N}: \forall n > N: a_n < b$$
	\item Если $$\lim_{n \to \infty} a_n = a$$ и $$a_n \leq b$$ то $$a \leq b$$ Аналогично для $\geq$.
	\item Если $$\exists \lim_{n \to \infty} a_n = a, \exists \lim_{n \to \infty} b_n = b$$ и $$a_n < b_n \text{ или } a_n \leq b_n$$ то $$a \leq b$$ 
	\end{enumerate}
	
	\section{Арифметические операции со сходящимися последовательностями.}

	\textbf{Утв.} Если $$\exists \lim_{n \to \infty} a_n = a, \exists \lim_{n \to \infty} b_n = b$$ то 
	\begin{enumerate}
	\item $$\exists \lim_{n \to \infty} a_n \pm b_n = a \pm` b$$
	\item $$\exists \lim_{n \to \infty} a_n b_n = a b$$
	\item Если дополнительно $b_n \neq 0$ $$\exists \lim_{n \to \infty} \dfrac{a_n}{b_n} = \dfrac{a}{b}$$
	\end{enumerate}
	
	\section{Теорема о пределе ограниченной монотонной последовательности.}

	\textbf{Утв.} Если $\{x_n\}$ - ограничена сверху (снизу) и монотонно возрастает (убывает), то $$\exists \lim_{n \to \infty} x_n = sup\{x_n\} (inf\{x_n\})$$

	\section{Число e.}
	
	\textbf{Утв.} $$\exists \lim_{n \to \infty} (1 + \dfrac{1}{n})^n = e$$

	\section{Теорема Кантора о вложенных отрезках.}

	\textbf{Утв.} У системы вложенных отрезков $[a_1;b_1] \supset [a_2; b_2] \supset [a_3; b_3] \cdots$ существует общая точка, причём единственная. 
	
	\section{Подпоследовательности и частичные пределы. Теорема о трёх определениях верхнего и нижнего пределов.}

	\textbf{Опр.} Рассмотрим строго возрастающую последовательность последовательность $\{n_k\}: n_k \in \mathbb{N}$, тогда последовательность $\{x_{n_k}\}$ называется подпоследовательностью последовательности $\{x_n\}$.
	
	\textbf{Опр.} Если $\{x_{n_k}\}$ - подпоследовательность $\{x_n\}$, и $$\exists \lim_{k \to \infty} x_{n_k} = l \in \bar{\mathbb{R}}$$ то $l$ называется частичным пределом $\{x_n\}$.

	\textbf{Опр.} $$\lim_{k \to \infty} sup \{x_n\} (n >k) \text{ и } \lim_{k \to 
\infty} inf \{x_n\} (n > k)$$ называются верхним и нижним пределом $\{x_n\}$

	\textbf{Утв.} У любой последовательности существуют верхний и нижний пределы в $\bar{\mathbb{R}}$
	\section{Теорема Больцано-Вейерштрасса.}

	\textbf{Утв.} У любой ограниченной последовательности существует сходящаяся подпоследовательность.
	
	\section{Критерий Коши сходимости числовой последовательности.}

	\textbf{Опр.} Последовательность называется фундаментальной, если $$(\forall \varepsilon > 0) (\exists N \in \mathbb{N}) (\forall n > N) (\forall p \in \mathbb{N}) |x_{n + p} - x_n| < E$$

	\textbf{Утв.} Последовательность сходится тогда и только тогда, когда она фундаментальна.
	
	\section{Определение предела в точке по Коши и по Гейне. Их эквивалентность.}

	\textbf{Опр.} $l$ называется пределом фунции $f$ в точке $x_0$ (По Коши) если $$(\forall \varepsilon > 0) (\exists \delta > 0) (\forall x: 0 < |x - x_0| < \delta) |f(x) - l| < \varepsilon$$
	
	\textbf{Опр.} $l$ называется пределом функции $f$ в точке $x_0$ (По Гейне) если$$(\forall \{x_n\}: \lim_{n \to \infty} x_n = x_0, x_n \neq x_0) \lim_{n \to \infty} f(x_n) = l$$
	
	\textbf{Утв.} Данные определения эквивалентны.
	\section{Критейрий коши существования предела функции.}

	a
	
	\section{Существование односторонних пределов у монотонных функций.}

	a
	
	\section{Непрерывность функции в точке. Непрерывность сложной функции.}

	a
	
	\section{Ограниченность функции, непрерывной на отрезке.}

	a
	
	\section{Достижение (точной) верхней и (точной) нижней граней функцией, непрерывной на отрезке.}

	a
	
	\section{Теорема о промежуточных значениях непрерывной функции.}

	a
	
	\section{Теорема об обратной функции.}

	a

\end{flushleft}
\end{document}

